\documentclass[oneside]{article}
% Copyright (c) 2018 Michael Heilmann. All rights reserved.

\usepackage[utf8]{inputenc}
\usepackage{textcomp} % for \textlangle and \textrangle macros
\usepackage{xcolor}
%
\usepackage{tabularx} % for tabularx environment
%
\usepackage{lmodern}
\usepackage{microtype}
%
\usepackage{graphicx}
\usepackage[colorlinks]{hyperref}
\usepackage[margin=0.75in]{geometry}
\usepackage{amsmath,amssymb,amsfonts}

\makeatletter

% "(Get|Set)Author".
% The name of the author.
\def\SetAuthor#1{\gdef\@author{#1}}
\def\@author{\@latex@warning@no@line{No author given}}
\def\GetAuthor{\@author}

% "(Get|Set)Email".
% The email (address) of the author.
\def\SetEmail#1{\gdef\@email{#1}}
\def\@email{\@latex@warning@no@line{No email given}}
\def\GetEmail{\@email}

% "(Get|Set)Organization".
% The organization the library is published by.
\def\SetOrganization#1{\gdef\@organization{#1}}
\def\@organization{\@latex@warning@no@line{No organization given}}
\def\GetOrganization{\@organization}

% "(Get|Set)LibraryName".
% The name of the library.
\def\SetLibraryName#1{\gdef\@libraryName{#1}}
\def\@libraryName{\@latex@warning@no@line{No library name given}}
\def\GetLibraryName{\@libraryName}

% "(Get|Set)LibraryVersion".
% The version of the library.
\def\SetLibraryVersion#1{\gdef\@libraryVersion{#1}}
\def\@libraryVersion{\@latex@warning@no@line{No library version given}}
\def\GetLibraryVersion{\@libraryVersion}

% "(Get|Set)LibraryRepository".
% The repository (url) of the library.
\def\SetLibraryRepository#1{\gdef\@libraryRepository{#1}}
\def\@libraryrepository{\@latex@warning@no@line{No library repository given}}
\def\GetLibraryRepository{\@libraryRepository}

% Define "maketitle".
\def\maketitle{%
  \noindent\makebox[\textwidth]{%
	\uppercase{{\GetOrganization} {\GetLibraryName} {\GetLibraryVersion}}%
	\hfill
	\uppercase{{\GetAuthor} {(\href{mailto:\GetEmail}{\GetEmail})}}%
  }%
}

\makeatother


\SetOrganization{Primordial Machine's}
\SetLibraryName{Trigonometry Library}
\SetLibraryVersion{v0.1}
\SetLibraryRepository{https://github.com/primordialmachine/trigonometry}
\SetAuthor{Michael Heilmann}
\SetEmail{michaelheilmann@primordialmachine.com}

\begin{document}
\maketitle
\tableofcontents
\section{Synopsis}
C++ 17 library providing functionality related to trigonometry.
The library is made available publicly on
\href{\GetLibraryRepository}{Github}
under the
\href{\GetLibraryRepository/blob/master/LICENSE}{MIT License}.

\section{Requirements}
If the library is added to a project, then one needs to add Primordial Machine libraries
\href{https://github.com/primordialmachine/one-zero-functors}{One Zero Functors Library}
as well.

\section{Limitations and Restrictions}
The library officially only supports Visual Studio 2017 and Windows 10.
%The library provides support for floating point types i.e. types which are provided by the C++ language.
%However, as a proof of concept of the abstractions, fixed point and arbitrary precision implementations need to be added to the library.

%\section{Introductory example}
%Examples are located in the \href{\GetLibraryRepository/blob/master/examples}{examples} directory.

\section{Building under Visual Studio 2017}
\begin{enumerate}
\item Open the solution \texttt{solution.sln} in Microsoft Visual Studio 2017.
\item Batch build everything.
\item The folder \texttt{packages} contains the distribution of the library i.e. include files and the
      static libraries for
  \begin{enumerate}
    \item the platforms \texttt{Win32} and \texttt{x64} and
    \item configurations \texttt{Release} and \texttt{Debug}.
  \end{enumerate}
\item Copy the contents of the \verb+packages+ folder into a directory. Let
      \verb+[library home]+ be a placeholder denoting the path by which that folder
      can be referenced from your project.
\item Add
  \begin{enumerate}
    \item the include path
\begin{verbatim}
[library home]/primordialmachine/trigonometry/
$(PlatformTarget.toLower())/$(Configuration.toLower())/includes
\end{verbatim}
	and
    \item the library path
\begin{verbatim}
[library home]/primordialmachine/trigonometry/
$(PlatformTarget.toLower())/$(Configuration.toLower())/libraries
\end{verbatim}
    to your project.
\end{enumerate}
\item Link your project with the library \verb+trigonometry.lib+.
\item Add the include directive \verb+#include "primordialmachine/trigonometry/include.hpp"+ where appropriate.
\item You can now use the functionality provided by the library.
\end{enumerate}

\section{Library Interface Documentation}

\subsection{\texttt{namespace primordialmachine}}
The namespace this library is adding its declarations/definitions to.
The added namespace elements are documented below.

\subsection{\textit{concept NullaryFunctor}}
A \textit{NullaryFunctor} is a function object with two template arguments \texttt{RESULT} and
\texttt{ENABLED}. Specializations of this type must provide a constant \verb+operator()+ with
zero parameters. The return value of that operand is a value of type \texttt{RESULT} (or constant
reference to a value of that type).\\

\subsection{\textit{concept UnaryFunctor}}
A \textit{NullaryFunctor} is a function object with three template arguments \texttt{RESULT},
\textit{PARAMETER}, and \texttt{ENABLED}. Specializations of this type must provide a constant
\verb+operator()+ with one parameters of type \textit{PARAMETER} (or a cv-qualified variant of
that). The return value of that operand is a value of type \texttt{RESULT} (or constant
reference to a value of that type).\\

\noindent{}The default type of \verb+ENABLED+ is \verb+void+. Specializations of this type may use
the \verb+ENABLED+ argument to perform \textit{SFINAE}.\\

\noindent{}\textcolor{orange}{\textit{Defect: No error/exception specification is provided.}}

\subsection{\texttt{struct pi\_functor}}
A \textit{NullaryFunctor} which provides (an approximation of) the value $\pi$.
Specializations of for the types \texttt{float}, \texttt{double}, and \texttt{long double} are provided.

\noindent{}Usage example:
\begin{verbatim}
#include "primordialmachine/trigonometry/include.hpp"
#include <stdlib.h>
#include <iostream>

int main(int argc, char **argv) {
  std::cout << primordialmachine::pi_functor<float>()() << std::endl;
  std::cout << primordialmachine::pi_functor<double>()() << std::endl;
  std::cout << primordialmachine::pi_functor<long double>()() << std::endl;  
  return EXIT_SUCCESS;
}
\end{verbatim}

\subsection{\texttt{pi}}
A function which returns the value of \texttt{primordialmachine::pi\_functor\textlangle T\textrangle}
for the type \texttt{T}.

\noindent{}Possible implementations
\begin{verbatim}
template<typename T>
auto pi() -> decltype(pi_functor<T>()())
{ return pi_functor<T>()(); }
\end{verbatim}

\subsection{\textit{concept AngleUnit}}
An \textit{AngleUnit} is a \texttt{struct} type denoting a unit in which an angle is measured.
This library provides types corresponding to the angle unit \textit{degrees}, \textit{radians},
and \textit{turns}.

\subsection{\texttt{angle\_unit\_degrees}}
A \textit{AngleUnit} denoting the angle unit \textit{degrees}.
\subsection{\texttt{angle\_unit\_radians}}
A \textit{AngleUnit} denoting the angle unit \textit{radians}.
\subsection{\texttt{angle\_unit\_turns}}
A \textit{AngleUnit} denoting the angle unit \textit{turns}.

%%%%%%%%%%%%%%%%%%%%%%%%%%%%%%%%%%%%%%%%%%%%%%%%%%%%%%%%%%%%%%%%%%%%%%%%%%%%%%%%%%%%%%%%%%%%%%%%%%%%
\subsection{\texttt{sin\_functor}}
A \textit{UnaryFunctor} which computes the
sine
of an angle.
The first parameter is the angle, the return value the sine of that angle.

\noindent{}This library provides specializations of this functor for all floating point types.
The specializations assume the angle is measured in radians.\\

\noindent{}\textcolor{orange}{\textit{Defect: No error/exception specification is provided.}}

\subsection{\texttt{sin}}
A function which returns the value of \texttt{primordialmachine::sin\_functor\textlangle T\textrangle}
for a value of type \texttt{T}.

\noindent{}Possible implementations
\begin{verbatim}
template<typename T>
auto sin(const T& v) -> decltype(sin_functor<T, void>()(v))
{ return sin_functor<T, void>()(v); }
\end{verbatim}

\noindent{}\textcolor{orange}{\textit{Defect: No error/exception specification is provided.}}

%%%%%%%%%%%%%%%%%%%%%%%%%%%%%%%%%%%%%%%%%%%%%%%%%%%%%%%%%%%%%%%%%%%%%%%%%%%%%%%%%%%%%%%%%%%%%%%%%%%%
\subsection{\texttt{cos\_functor}}
A \textit{UnaryFunctor} which computes the
cosine
of an angle.
The first parameter is the angle, the return value the cosine of that angle.

\noindent{}This library provides specializations of this functor for all floating point types.
The specializations assume the angle is measured in radians.\\

\noindent{}\textcolor{orange}{\textit{Defect: No error/exception specification is provided.}}

\subsection{\texttt{cos}}
A function which returns the value of \texttt{primordialmachine::cos\_functor\textlangle T\textrangle}
for a value of type \texttt{T}.

\noindent{}Possible implementations
\begin{verbatim}
template<typename T>
auto cos(const T& v) -> decltype(cos_functor<T, void>()(v))
{ return cos_functor<T, void>()(v); }
\end{verbatim}

\noindent{}\textcolor{orange}{\textit{Defect: No error/exception specification is provided.}}

%%%%%%%%%%%%%%%%%%%%%%%%%%%%%%%%%%%%%%%%%%%%%%%%%%%%%%%%%%%%%%%%%%%%%%%%%%%%%%%%%%%%%%%%%%%%%%%%%%%%
\subsection{\texttt{tan\_functor}}
A \textit{UnaryFunctor} which computes the
tangens
of an angle.
The first parameter is the angle, the return value the tangens of that angle.

\noindent{}This library provides specializations of this functor for all floating point types.
The specializations assume the angle is measured in radians.\\

\noindent{}\textcolor{orange}{\textit{Defect: No error/exception specification is provided.}}

\subsection{\texttt{tan}}
A function which returns the value of \texttt{primordialmachine::tan\_functor\textlangle T\textrangle}
for a value of type \texttt{T}.

\noindent{}A possible implementation is
\begin{verbatim}
template<typename T>
auto tan(const T& v) -> decltype(tan_functor<T, void>()(v))
{ return tan_functor<T, void>()(v); }
\end{verbatim}

\noindent{}\textcolor{orange}{\textit{Defect: No error/exception specification is provided.}}

%%%%%%%%%%%%%%%%%%%%%%%%%%%%%%%%%%%%%%%%%%%%%%%%%%%%%%%%%%%%%%%%%%%%%%%%%%%%%%%%%%%%%%%%%%%%%%%%%%%%
\subsection{\texttt{acos\_functor}}
A \textit{UnaryFunctor} which computes the
arccosine
of a value.
The first parameter is the value, the return value the arccosine of that value.

\noindent{}This library provides specializations of this functor for all floating point types.
For these specializations, \texttt{RESULT} is of type \texttt{PARAMETER} and the return value
denotes the inverse trigonometric function value for the specified input value.\\

\noindent{}\textcolor{orange}{\textit{Defect: No error/exception specification is provided.}}

\subsection{\texttt{acos}}
A function which returns the value of \texttt{primordialmachine::acos\_functor\textlangle T\textrangle}
for a value of type \texttt{T}.

\noindent{}A possible implementation is
\begin{verbatim}
template<typename T>
auto acos(const T& v) -> decltype(acos_functor<T, void>()(v))
{ return acos_functor<T, void>()(v); }
\end{verbatim}

\noindent{}\textcolor{orange}{\textit{Defect: No error/exception specification is provided.}}
%%%%%%%%%%%%%%%%%%%%%%%%%%%%%%%%%%%%%%%%%%%%%%%%%%%%%%%%%%%%%%%%%%%%%%%%%%%%%%%%%%%%%%%%%%%%%%%%%%%%
\subsection{\texttt{asin\_functor}}
A \textit{UnaryFunctor} which computes the
arcsine
of a value.
The first parameter is the value, the return value the arcsine of that value.

\noindent{}This library provides specializations of this functor for all floating point types.
For these specializations, \texttt{RESULT} is of type \texttt{PARAMETER} and the return value
denotes the inverse trigonometric function value for the specified input value.\\

\noindent{}\textcolor{orange}{\textit{Defect: No error/exception specification is provided.}}

\subsection{\texttt{asin}}
A function which returns the value of \texttt{primordialmachine::asin\_functor\textlangle T\textrangle}
for a value of type \texttt{T}.

\noindent{}A possible implementation is
\begin{verbatim}
template<typename T>
auto asin(const T& v) -> decltype(asin_functor<T, void>()(v))
{ return asin_functor<T, void>()(v); }
\end{verbatim}

\noindent{}\textcolor{orange}{\textit{Defect: No error/exception specification is provided.}}
%%%%%%%%%%%%%%%%%%%%%%%%%%%%%%%%%%%%%%%%%%%%%%%%%%%%%%%%%%%%%%%%%%%%%%%%%%%%%%%%%%%%%%%%%%%%%%%%%%%%
\subsection{\texttt{cot\_functor}}
A \textit{UnaryFunctor} which computes the
cotangens of an angle.
The first parameter is the angle, the return value the cotangens of that angle.

\noindent{}This library provides specializations of this functor for all floating point types.
The specializations assume the angle is measured in radians.\\

\noindent{}\textcolor{orange}{\textit{Defect: No error/exception specification is provided.}}

\subsection{\texttt{cot}}
A function which returns the value of \texttt{primordialmachine::cot\_functor\textlangle T\textrangle}
for a value of type \texttt{T}.

\noindent{}A possible implementation is
\begin{verbatim}
template<typename T>
auto cot(const T& v) -> decltype(cot_functor<T, void>()(v))
{ return cot_functor<T, void>()(v); }
\end{verbatim}

\noindent{}\textcolor{orange}{\textit{Defect: No error/exception specification is provided.}}
%%%%%%%%%%%%%%%%%%%%%%%%%%%%%%%%%%%%%%%%%%%%%%%%%%%%%%%%%%%%%%%%%%%%%%%%%%%%%%%%%%%%%%%%%%%%%%%%%%%%
\end{document}
