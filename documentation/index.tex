%%%%%%%%%%%%%%%%%%%%%%%%%%%%%%%%%%%%%%%%%%%%%%%%%%%%%%%%%%%%%%%%%%%%%%%%%%%%%%%%%%%%%%%%%%%%%%%%%%%
%
% Primordial Machine's Math Trigonometry Library
% Copyright (C) 2017-2019 Michael Heilmann
%
% This software is provided 'as-is', without any express or implied warranty.
% In no event will the authors be held liable for any damages arising from the
% use of this software.
%
% Permission is granted to anyone to use this software for any purpose,
% including commercial applications, and to alter it and redistribute it
% freely, subject to the following restrictions:
%
% 1. The origin of this software must not be misrepresented;
%    you must not claim that you wrote the original software.
%    If you use this software in a product, an acknowledgment
%    in the product documentation would be appreciated but is not required.
%
% 2. Altered source versions must be plainly marked as such,
%    and must not be misrepresented as being the original software.
%
% 3. This notice may not be removed or altered from any source distribution.
%
%%%%%%%%%%%%%%%%%%%%%%%%%%%%%%%%%%%%%%%%%%%%%%%%%%%%%%%%%%%%%%%%%%%%%%%%%%%%%%%%%%%%%%%%%%%%%%%%%%%

\documentclass[oneside]{book}

\input{common}

\SetOrganization{Primordial Machine's}
\SetLibraryName{Math Trigonometry Library}

\SetLibraryIncludeFileName{include.hpp}
\SetLibraryIncludesDirectoryPath{primordialmachine/math-trigonometry/\newline\$(PlatformTarget.toLower())/\$(Configuration.toLower())/includes}

\SetLibraryIncludeDirectiveFilePath{primordialmachine/math/trigonometry/include.hpp}

\SetLibraryStaticLibrariesDirectoryPath{primordialmachine/math-trigonometry/\newline\$(PlatformTarget.toLower())/\$(Configuration.toLower())/libraries}
\SetLibraryStaticLibraryFileName{math-trigonometry.lib}

\SetLibraryVersion{v1.3}
\SetLibraryRepository{https://github.com/primordialmachine/math-trigonometry}
\SetAuthor{Michael Heilmann}
\SetEmail{michaelheilmann@primordialmachine.com}

\SetDocumentType{User Manual}

\begin{document}

\frontmatter

\begin{titlepage}
\maketitle
\end{titlepage}

\tableofcontents
\addtocontents{toc}{\protect\thispagestyle{empty}}
\pagenumbering{gobble}

\mainmatter

\chapter{Synopsis}
C++ 17 library providing functionality related to trigonometry.
The library is made available publicly on
\href{\GetLibraryRepository}{Github}
under the
\href{\GetLibraryRepository/blob/master/LICENSE}{MIT License}.

\chapter{Requirements}
If the library is added to a project, then one needs to add Primordial Machine libraries
\href{https://github.com/primordialmachine/one-zero-functors}{One Zero Functors Library}
as well.

\chapter{Limitations and Restrictions}
The library officially only supports Visual Studio 2017 and Windows 10.
%The library provides support for floating point types i.e. types which are provided by the C++ language.
%However, as a proof of concept of the abstractions, fixed point and arbitrary precision implementations need to be added to the library.

%\section{Introductory example}
%Examples are located in the \href{\GetLibraryRepository/blob/master/examples}{examples} directory.

\input{building_visual_studio_2017}

\chapter{Library Interface Documentation}

\section{\texttt{namespace primordialmachine}}
The namespace this library is adding its declarations/definitions to.
The added namespace elements are documented below.

\section{\textit{concept NullaryFunctor}}
A \textit{NullaryFunctor} is a function object with two template arguments \texttt{RESULT} and
\texttt{ENABLED}. Specializations of this type must provide a constant \verb+operator()+ with
zero parameters. The return value of that operand is a value of type \texttt{RESULT} (or constant
reference to a value of that type).\\

\section{\textit{concept UnaryFunctor}}
A \textit{NullaryFunctor} is a function object with three template arguments \texttt{RESULT},
\textit{PARAMETER}, and \texttt{ENABLED}. Specializations of this type must provide a constant
\verb+operator()+ with one parameters of type \textit{PARAMETER} (or a cv-qualified variant of
that). The return value of that operand is a value of type \texttt{RESULT} (or constant
reference to a value of that type).\\

\noindent{}The default type of \verb+ENABLED+ is \verb+void+. Specializations of this type may use
the \verb+ENABLED+ argument to perform \textit{SFINAE}.\\

\noindent{}\textcolor{orange}{\textit{Defect: No error/exception specification is provided.}}

\section{\texttt{struct pi\_functor}}
A \textit{NullaryFunctor} which provides (an approximation of) the value $\pi$.
Specializations of for the types \texttt{float}, \texttt{double}, and \texttt{long double} are provided.

\noindent{}Usage example:
\begin{verbatim}
#include "primordialmachine/trigonometry/include.hpp"
#include <stdlib.h>
#include <iostream>

int main(int argc, char **argv) {
  using namspace primordialmachine;
  std::cout << pi_functor<float>()() << std::endl;
  std::cout << pi_functor<double>()() << std::endl;
  std::cout << pi_functor<long double>()() << std::endl;  
  return EXIT_SUCCESS;
}
\end{verbatim}

\subsection{\texttt{pi}}
A function which returns the value of \texttt{primordialmachine::pi\_functor\textlangle T\textrangle}
for the type \texttt{T}.

\noindent{}Possible implementations
\begin{verbatim}
template<typename T>
auto pi() -> decltype(pi_functor<T>()())
{ return pi_functor<T>()(); }
\end{verbatim}

\section{\textit{concept AngleUnit}}
An \textit{AngleUnit} is a \texttt{struct} type denoting a unit in which an angle is measured.
This library provides types corresponding to the angle unit \textit{degrees}, \textit{radians},
and \textit{turns}.

\subsection{\texttt{angle\_unit\_degrees}}
A \textit{AngleUnit} denoting the angle unit \textit{degrees}.
\subsection{\texttt{angle\_unit\_radians}}
A \textit{AngleUnit} denoting the angle unit \textit{radians}.
\subsection{\texttt{angle\_unit\_turns}}
A \textit{AngleUnit} denoting the angle unit \textit{turns}.

%%%%%%%%%%%%%%%%%%%%%%%%%%%%%%%%%%%%%%%%%%%%%%%%%%%%%%%%%%%%%%%%%%%%%%%%%%%%%%%%%%%%%%%%%%%%%%%%%%%%
\section{\texttt{sin\_functor}}
A \textit{UnaryFunctor} which computes the
sine
of an angle.
The first parameter is the angle, the return value the sine of that angle.

\noindent{}This library provides specializations of this functor for all floating point types.
The specializations assume the angle is measured in radians.\\

\noindent{}\textcolor{orange}{\textit{Defect: No error/exception specification is provided.}}

\subsection{\texttt{sin}}
A function which returns the value of \texttt{primordialmachine::sin\_functor\textlangle T\textrangle}
for a value of type \texttt{T}.

\noindent{}Possible implementations
\begin{verbatim}
template<typename T>
auto sin(const T& v) -> decltype(sin_functor<T, void>()(v))
{ return sin_functor<T, void>()(v); }
\end{verbatim}

\noindent{}\textcolor{orange}{\textit{Defect: No error/exception specification is provided.}}

%%%%%%%%%%%%%%%%%%%%%%%%%%%%%%%%%%%%%%%%%%%%%%%%%%%%%%%%%%%%%%%%%%%%%%%%%%%%%%%%%%%%%%%%%%%%%%%%%%%%
\section{\texttt{cos\_functor}}
A \textit{UnaryFunctor} which computes the
cosine
of an angle.
The first parameter is the angle, the return value the cosine of that angle.

\noindent{}This library provides specializations of this functor for all floating point types.
The specializations assume the angle is measured in radians.\\

\noindent{}\textcolor{orange}{\textit{Defect: No error/exception specification is provided.}}

\subsection{\texttt{cos}}
A function which returns the value of \texttt{primordialmachine::cos\_functor\textlangle T\textrangle}
for a value of type \texttt{T}.

\noindent{}Possible implementations
\begin{verbatim}
template<typename T>
auto cos(const T& v) -> decltype(cos_functor<T, void>()(v))
{ return cos_functor<T, void>()(v); }
\end{verbatim}

\noindent{}\textcolor{orange}{\textit{Defect: No error/exception specification is provided.}}

%%%%%%%%%%%%%%%%%%%%%%%%%%%%%%%%%%%%%%%%%%%%%%%%%%%%%%%%%%%%%%%%%%%%%%%%%%%%%%%%%%%%%%%%%%%%%%%%%%%%
\section{\texttt{tan\_functor}}
A \textit{UnaryFunctor} which computes the
tangens
of an angle.
The first parameter is the angle, the return value the tangens of that angle.

\noindent{}This library provides specializations of this functor for all floating point types.
The specializations assume the angle is measured in radians.\\

\noindent{}\textcolor{orange}{\textit{Defect: No error/exception specification is provided.}}

\subsection{\texttt{tan}}
A function which returns the value of \texttt{primordialmachine::tan\_functor\textlangle T\textrangle}
for a value of type \texttt{T}.

\noindent{}A possible implementation is
\begin{verbatim}
template<typename T>
auto tan(const T& v) -> decltype(tan_functor<T, void>()(v))
{ return tan_functor<T, void>()(v); }
\end{verbatim}

\noindent{}\textcolor{orange}{\textit{Defect: No error/exception specification is provided.}}

%%%%%%%%%%%%%%%%%%%%%%%%%%%%%%%%%%%%%%%%%%%%%%%%%%%%%%%%%%%%%%%%%%%%%%%%%%%%%%%%%%%%%%%%%%%%%%%%%%%%
\section{\texttt{acos\_functor}}
A \textit{UnaryFunctor} which computes the
arccosine
of a value.
The first parameter is the value, the return value the arccosine of that value.

\noindent{}This library provides specializations of this functor for all floating point types.
For these specializations, \texttt{RESULT} is of type \texttt{PARAMETER} and the return value
denotes the inverse trigonometric function value for the specified input value.\\

\noindent{}\textcolor{orange}{\textit{Defect: No error/exception specification is provided.}}

\subsection{\texttt{acos}}
A function which returns the value of \texttt{primordialmachine::acos\_functor\textlangle T\textrangle}
for a value of type \texttt{T}.

\noindent{}A possible implementation is
\begin{verbatim}
template<typename T>
auto acos(const T& v) -> decltype(acos_functor<T, void>()(v))
{ return acos_functor<T, void>()(v); }
\end{verbatim}

\noindent{}\textcolor{orange}{\textit{Defect: No error/exception specification is provided.}}
%%%%%%%%%%%%%%%%%%%%%%%%%%%%%%%%%%%%%%%%%%%%%%%%%%%%%%%%%%%%%%%%%%%%%%%%%%%%%%%%%%%%%%%%%%%%%%%%%%%%
\section{\texttt{asin\_functor}}
A \textit{UnaryFunctor} which computes the
arcsine
of a value.
The first parameter is the value, the return value the arcsine of that value.

\noindent{}This library provides specializations of this functor for all floating point types.
For these specializations, \texttt{RESULT} is of type \texttt{PARAMETER} and the return value
denotes the inverse trigonometric function value for the specified input value.\\

\noindent{}\textcolor{orange}{\textit{Defect: No error/exception specification is provided.}}

\subsection{\texttt{asin}}
A function which returns the value of \texttt{primordialmachine::asin\_functor\textlangle T\textrangle}
for a value of type \texttt{T}.

\noindent{}A possible implementation is
\begin{verbatim}
template<typename T>
auto asin(const T& v) -> decltype(asin_functor<T, void>()(v))
{ return asin_functor<T, void>()(v); }
\end{verbatim}

\noindent{}\textcolor{orange}{\textit{Defect: No error/exception specification is provided.}}
%%%%%%%%%%%%%%%%%%%%%%%%%%%%%%%%%%%%%%%%%%%%%%%%%%%%%%%%%%%%%%%%%%%%%%%%%%%%%%%%%%%%%%%%%%%%%%%%%%%%
\section{\texttt{cot\_functor}}
A \textit{UnaryFunctor} which computes the
cotangens of an angle.
The first parameter is the angle, the return value the cotangens of that angle.

\noindent{}This library provides specializations of this functor for all floating point types.
The specializations assume the angle is measured in radians.\\

\noindent{}\textcolor{orange}{\textit{Defect: No error/exception specification is provided.}}

\subsection{\texttt{cot}}
A function which returns the value of \texttt{primordialmachine::cot\_functor\textlangle T\textrangle}
for a value of type \texttt{T}.

\noindent{}A possible implementation is
\begin{verbatim}
template<typename T>
auto cot(const T& v) -> decltype(cot_functor<T, void>()(v))
{ return cot_functor<T, void>()(v); }
\end{verbatim}

\noindent{}\textcolor{orange}{\textit{Defect: No error/exception specification is provided.}}
%%%%%%%%%%%%%%%%%%%%%%%%%%%%%%%%%%%%%%%%%%%%%%%%%%%%%%%%%%%%%%%%%%%%%%%%%%%%%%%%%%%%%%%%%%%%%%%%%%%%
\section{\texttt{angle}}
Representation of an angle. \texttt{angle} represents the angle as a value of type \texttt{UNDERLYING\_TYPE}.
Furthermore, it denotes the unit in which the angle is measured by the type \texttt{UNIT}.
\begin{verbatim}
template<typename UNIT, typename UNDERLYING_TYPE, typename ENABLED = void>
struct angle;
\end{verbatim}

\begin{center}
\begin{tabularx}{\textwidth}{|>{\hsize=0.5\hsize}X|>{\hsize=1.5\hsize}X|}
\hline
\texttt{UNIT}             & Any \textit{AngleUnit} type denoting the unit in which the angle is measured.\\\hline
\texttt{UNDERLYING\_TYPE} & The underlying type used to represent the angle value.
                            \texttt{UNDERLYING\_TYPE} must be an \href{https://en.cppreference.com/w/c/language/arithmetic_types}{Arithmetic Type}.\\\hline
\texttt{ENABLED}          & For SFINAE. Default value is \texttt{void}.\\\hline
\end{tabularx}
\end{center}

\noindent{}The library provides specializations:
\begin{center}
\begin{tabularx}{\textwidth}{|X|X|}
\hline
\texttt{UNIT}             & Any \textit{AngleUnit} type provided by this library.\\\hline
\texttt{UNDERLYING\_TYPE} & \texttt{float}, \texttt{double}, or \texttt{long double}.\\\hline
\end{tabularx}
\end{center}

\subsection{Constructors}

\noindent{}\texttt{angle}\texttt{(}\texttt{)} (1)\\
\noindent{}\texttt{angle}\texttt{(}\texttt{UNDERLYING\_TYPE value}\texttt{)} (2)\\

\noindent{}(1) Construct the \texttt{angle} value with \texttt{zero\textlangle UNDERLYING\_TYPE\textrangle()}.\\
(2) Construct the \texttt{angle} value with the specified value \texttt{value}.\\

\subsection{Members}
\subsubsection{\texttt{operator+}}
Compute the sum of two \texttt{angle} values.\\

\noindent\texttt{angle\textlangle UNIT, UNDERLYING\_TYPE\textrangle}\\
\texttt{operator+}\texttt{(}\texttt{const angle\textlangle UNIT, UNDERLYING\_TYPE\textrangle\& a}\texttt{,}\\
\hphantom{\texttt{operator+}\texttt{(}}\texttt{const angle\textlangle UNIT, UNDERLYING\_TYPE\textrangle\& b)}\texttt{)}\\

\subsubsection{\texttt{operator-}}
Compute the difference of two angles.\\

\noindent\texttt{angle\textlangle UNIT, UNDERLYING\_TYPE\textrangle}\\
\texttt{operator-}\texttt{(}\texttt{const angle\textlangle UNIT, UNDERLYING\_TYPE\textrangle\& a}\texttt{,}\\
\hphantom{\texttt{operator-}\texttt{(}}\texttt{const angle\textlangle UNIT, UNDERLYING\_TYPE\textrangle\& b)}\texttt{)}\\

\subsubsection{\texttt{operator*}}
Compute the product of an angle and a scalar.\\

\noindent\texttt{angle\textlangle UNIT, UNDERLYING\_TYPE\textrangle}\\
\texttt{operator*}\texttt{(}\texttt{const angle\textlangle UNIT, UNDERLYING\_TYPE\textrangle\& a}\texttt{,}\\
\hphantom{\texttt{operator*}\texttt{(}}\texttt{UNDERLYING\_TYPE s}\texttt{)}\\

\noindent{}\texttt{angle\textlangle UNIT, UNDERLYING\_TYPE\textrangle}\\
\texttt{operator*}\texttt{(}\texttt{UNDERLYING\_TYPE s}\texttt{,}\\
\hphantom{\texttt{operator*}\texttt{(}}\texttt{const angle\textlangle UNIT, UNDERLYING\_TYPE\textrangle\& a}\texttt{)}\\

\subsubsection{\texttt{operator/}}
Compute the quotient of an angle and a scalar.\\

\noindent\texttt{angle\textlangle UNIT, UNDERLYING\_TYPE\textrangle}\\
\texttt{operator*}\texttt{(}\texttt{UNDERLYING\_TYPE s}\texttt{,}\\
\hphantom{\texttt{operator*}\texttt{(}}\texttt{const angle\textlangle UNIT, UNDERLYING\_TYPE\textrangle\& a)}\texttt{)}\\

\noindent\texttt{angle\textlangle UNIT, UNDERLYING\_TYPE\textrangle}\\
\texttt{operator/}\texttt{(}\texttt{const angle\textlangle UNIT, UNDERLYING\_TYPE\textrangle\& a}\texttt{,}\\
\hphantom{\texttt{operator*}\texttt{(}}\texttt{UNDERLYING\_TYPE s)}\\

\subsection{\texttt{to\_degrees}}
Get the angle in degrees.\\

\noindent\texttt{T}\\
\texttt{to\_degrees}\texttt{()} const\\

\noindent{}Returns a value of type $T$ where $T$ is either
\texttt{angle\textlangle angle\_unit\_degrees, UNDERLYING\_TYPE\textrangle}
or a reference to a constant value of that type.

\subsection{\texttt{to\_radians}}
Get the angle in radians.\\

\noindent\texttt{T}\\
\texttt{to\_radians}\texttt{()} const\\

\noindent{}Returns a value of type $T$ where $T$ is either
\texttt{angle\textlangle angle\_unit\_radians, UNDERLYING\_TYPE\textrangle}
or a reference to a constant value of that type.

\subsection{\texttt{to\_turns}}
Get the angle in turns.\\

\noindent\texttt{T}\\
\texttt{to\_turns}\texttt{()} const\\

\noindent{}Returns a value of type $T$ where $T$ is either
\texttt{angle\textlangle angle\_unit\_turns, UNDERLYING\_TYPE\textrangle}
or a reference to a constant value of that type.

\subsection{\texttt{value}}
Get the value of thisangle in turns.\\

\noindent\texttt{T}\\
\texttt{value() const} (1)\\

\noindent\texttt{T}\\
\texttt{value(T v)} (2)\\

\noindent{}(1) Get the value of this \textit{angle} object. \texttt{T} is either a value of type
\texttt{UNDERLYING\_TYPE} or a reference to a constant value of that type.\\

\noindent{}(2) Set the value of this \textit{angle} object. \texttt{T} is either value of type
\texttt{UNDERLYING\_TYPE} or a reference to a constant value of that type.\\

\subsection{Non members}
\subsubsection{Angle Functors}
The library provides partial specializations of \texttt{cos\_functor}, \texttt{cot\_functor},
\texttt{sin\_functor}, and \texttt{tan\_functor} for every specialization of \texttt{angle}
provided by this library.

%%%%%%%%%%%%%%%%%%%%%%%%%%%%%%%%%%%%%%%%%%%%%%%%%%%%%%%%%%%%%%%%%%%%%%%%%%%%%%%%%%%%%%%%%%%%%%%%%%%%
\end{document}
